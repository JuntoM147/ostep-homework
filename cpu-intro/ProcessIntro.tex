\documentclass[12pt]{article}
\title{\textbf{Process Intro Notes}}
\date{}

\usepackage{titling}
\setlength{\droptitle}{-8em}

\usepackage[most]{tcolorbox}

\newtcolorbox{definitionbox}[1][]{
  colback=blue!5!white,
  colframe=blue!75!black,
  fonttitle=\bfseries,
  title=Defintion,
  #1
}

\newtcolorbox{questionbox}[1][]{
  colback=white,
  colframe=red!75!black,
  fonttitle=\bfseries,
  title=Defintion,
  #1
}

\begin{document}
\maketitle
\vspace{-8em}
\section{Content}

\subsection{Introduction}

This chapter is all about \textit{processes} (a running program). And how the OS virtualizes the CPU to create the illusion that there are a large number of CPUs on the system.
By virtualizing the CPU it allows the user to not have to concern about wether a CPU is going to be available when running a program, instead you just run it.\\


\begin{questionbox}[title=Question]
  \textit{"Although there are only a few physical CPUs available, how can the OS provide the illusion of a nearly endless supply of said CPUs?"}
\end{questionbox}


One technique that is used to create this illusion is by \textbf{time sharing}. 
Time sharing involves running one process, stopping it and switching to a different process and so on. 
A potential downside of this is performance, because the CPU must be shared.\\

Implementation of this requires \textbf{mechanisms} that lie on or close to the hardware such as a \textbf{context-switch.} And on top of these mecahnisms lie certain \textbf{policies} that are in place for example to decide which program to run next?
The seperation between low level mechanisms from high level policies is a example of modularity.
The mechanism answers the \textit{how} (how does the OS perform a context switch) while the policy answers the \textit{which} (which process should the OS run next?).

\begin{definitionbox}[title= Definition]
  \textbf{\textit{Mechanism:}} Low-level methods or protocols that implement a piece of functionality
\end{definitionbox}


\begin{definitionbox}[title= Definition]
  \textbf{\textit{Policies:}} Algorithms for making some kind of decision within the OS
\end{definitionbox}

\subsection{Process}
For the virtualization, the \textbf{machine state} is needed to determine what is important for the execution of the program. 
\begin{enumerate}
    \item Memory, specifically the \textbf{address space} (The memory that a process can address) of the program 
    \item Registers, including PC, stack pointer, frame pointer etc.
    \item I/O information, for example a list of the files the process has open 
\end{enumerate}

\begin{definitionbox}[title= Definition]
  \textbf{\textit{Machine state:}} What a program can read or update when it is running
\end{definitionbox}

\subsection{Process API}
A general idea of what a process API might look like is listed below.

\begin{itemize}
  \item \textbf{Create:} An OS must include some method to create a new process
  \item \textbf{Destroy:} OS must provide some way to destroy processes forcefully.
  \item \textbf{Wait:} Some way to wait for another process to exit.
  \item \textbf{Miscellaneous Control:}  Other controls such as suspension of a process and resuming it later.
  \item \textbf{Status:} A way to get the status information about a process, e.g. how long it's been running and the current state.
\end{itemize}

\subsection{Process Creation}
To create a process:

\begin{enumerate}
  \item Load the code and any \textbf{static data} into the address space of the process
  \item Some memory is allocated for the program's run time stack, including initialisation of stack argument like argc and argv for main() in C
  \item The OS may allocate some memory for the program's heap. (malloc and free) 
  \item I/O setup. In UNIX systems each process has 3 default \textbf{file descriptors} open: standard input, standard output, and standard error. 
  \item Start the program running at the entry point (jump to main()).
  \item Pass control of the CPU to the process and program begins execution.
\end{enumerate}

\subsection{Process States}
A simplified view of the possible states a process can be in is listed below.

\begin{itemize}
  \item \textbf{Running:} The process is running on a processor and executing instructions. 
  \item \textbf{Ready:} A process is ready to run but not chosen by the OS to run at this moment.
  \item \textbf{Blocked:} A process has performed some kind of operation that makes it not ready to run until some other event takes place. (e.g. I/O request)
\end{itemize}

When a process is moved from the ready to running it is said to have been \textit{scheduled}. In contrast, a process being moved from running to ready means it has been \textit{descheduled}. 

\subsection{Data Structures}

\section{Homework}
\end{document}
